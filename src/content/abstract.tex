\pagenumbering{roman}
\setcounter{page}{1}

\selecthungarian

%----------------------------------------------------------------------------
% Abstract in Hungarian
%----------------------------------------------------------------------------
\chapter*{Kivonat}\addcontentsline{toc}{chapter}{Kivonat}

A gerincproblémák diagnosztizálásának egyik leggyakrabban használt módszere a gerinc CT felvételek elemzése. A diagnózis felállítása során az orvosok több kulcsfontosságú tényezőt és jellemzőt vizsgálnak meg, hogy átfogó képet kapjanak a páciens állapotáról. Ilyen például a csontszerkezet integritásának vizsgálata. Ez a folyamat egyszerű diagnózisoknál is sok időt vehet igénybe, komplex esetekben (többszörös sérülés, korábbi műtétek ellenőrzése) pedig jelentősen megnőhet az elemzésével töltött idő. Trauma esetén gyakran sürgős beavatkozás szükséges, így a folyamat felgyorsítása életmentő lehet. 
A csigolyák mérete és alakja a gerinc mentén folyamatosan változik. A felvételek minősége és a páciensek egyedi genetikai jegyei miatt szakértői megoldások használata nehéz, jellemzően csak a végeredmény minőségének romlása mellett lehetséges. Emiatt a probléma megoldására egy gépi tanulásos módszert javaslok, amihez megfelelően annotált adathalmazok szükségesek. 
Több kétdimenziós szegmentáló módszer is elérhető elfogadható teljesítménnyel, ezek viszont nem tudják kihasználni a háromdimenziós CT felvételek térbeli információit. Az elérhető háromdimenziós modellek számítási igénye hatalmas, így az ilyen neurális hálózatok tanítása és használata magas költségekhez és hosszú futásidőhöz vezet.

Dolgozatomban a gerinc szegmentálására egy hibrid U-Net modellt vizsgálok meg, ami kétdimenziós előfeldolgozás segítségével redukálja az információt a háromdimenziós modell gyorsabb és pontosabb tanítása érdekében. Az implementálni kívánt eljárást a kutatási célokra elérhető VerSe 2019 adathalmazon értékelem ki, majd hasonlítom össze egy "state of the art" konvolúciós neurális hálózat eredményeivel.



\vfill
\selectenglish


%----------------------------------------------------------------------------
% Abstract in English
%----------------------------------------------------------------------------
\chapter*{Abstract}\addcontentsline{toc}{chapter}{Abstract}

One of the predominant approaches for diagnosing spinal problems is the analysis of spinal CT scans. When making a diagnosis, doctors examine several key factors and features to get a comprehensive picture of the patient's condition. For example, the integrity of the bone structure. This process can be time-consuming even for simple diagnoses, and in complex cases (multiple injuries, checking previous surgeries) the time spent on analysis can increase significantly. In the case of trauma, the patient may also need urgent intervention, so speeding up the process can be life-saving. 
The size and shape of the vertebrae along the spine are constantly changing. Due to the quality of the images and the unique genetic traits of the patients, expert solutions are difficult to use, typically only with a deterioration in the quality of the final result. For this reason, I propose a machine learning approach to solve this problem, which requires appropriately annotated datasets. 
Several two-dimensional segmentation methods are available with acceptable performance, but they cannot exploit the spatial information of three-dimensional CT images. The computational demand of the available three-dimensional models is huge, leading to high costs and long runtimes for training and using such neural networks.

In my thesis, I explore a specialized U-Net model designed for spine segmentation. This model employs a two-step process: first, it simplifies the data using two-dimensional preprocessing, and then it uses this simplified data to more quickly and accurately train a three-dimensional model. I evaluate the method on the VerSe 2019 dataset available for research purposes and compare it to the results of a state-of-the-art convolutional neural network used for biomedical image segmentation.


\vfill
\cleardoublepage

\selectthesislanguage

\newcounter{romanPage}
\setcounter{romanPage}{\value{page}}
\stepcounter{romanPage}